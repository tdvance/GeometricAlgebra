\documentclass{amsproc}
\usepackage{amssymb}
\usepackage{graphicx}
\usepackage[cmtip,all]{xy}

\copyrightinfo{2016}{Todd D. Vance}

\newtheorem{theorem}{Theorem}[section]
\newtheorem{lemma}[theorem]{Lemma}

\theoremstyle{definition}
\newtheorem{definition}[theorem]{Definition}
\newtheorem{example}[theorem]{Example}
\newtheorem{xca}[theorem]{Exercise}

\theoremstyle{remark}
\newtheorem{remark}[theorem]{Remark}

\numberwithin{equation}{section}

\begin{document}
\title{A Geometric Algebra Primer}
\author[T. D. Vance]{Todd D. Vance}
%\address{15 Lilac Dr \\ Moorefield, WV 26836}
%\email{tdvance@gmail.com}
\date{\today}

\begin{abstract}
  A brief introduction to Geometric Algebra with an emphasis on algorithms.
\end{abstract}

\maketitle{}

\newcommand{\term}[1]{\emph{#1}}

\newcommand{\ga}{\ensuremath{\mathop{\mathrm{GA}}}}

\section{definitions}

\begin{definition}[Geometric Algebra]
  The \term{geometric algebra} $\ga(n)$ of \term{ambient rank} $n$ is
  the quotient algebra $\mathbf{R}\{[1], [2], \dots, [n]\}/I$, where
  $\mathbf{R}\{[1], [2], \dots, [n]\}$ is the free (unital,
  associative, noncommutative) algebra over the real field
  $\mathbf{R}$ generated by noncommuting elements $[1], [2], \dots,
  [n]$, and $I$ is the (two-sided) ideal of $\mathbf{R}\{[1], [2],
  \dots, [n]\}$ generated by the set of elements of the form $[i][i] -
  1$ and of the form $[i][j] - [j][i]$ when $i\ne{j}$.
\end{definition}

The multiplication operator in $\ga(n)$ is called the \term{geometric
  product} and is written by juxtaposition: $ab$ is the product of $a$
and $b$, in that order.  An (equivalence class of the) element of
$\ga(n)$ of the form $[i_1][i_2]\cdots[i_m]$ is abbreviated
$[i_1i_2\cdots{i}_m]$ and is called a \term{basis blade}.  As a
special case, the empty basis blade $[]$ is defined to be the
multiplicative identity $1$.

\begin{example}
A commonly-used geometric algebra in graphics applications is $\ga[3]$
of ambient rank $3$.  The algebra is generated by the elements $[1]$,
$[2]$, and $[3]$.  Clearly $[1][1] = [] = 1$, $[1][2]=-[2][1]$, and so
on.  From this one finds that $\ga(3)$ is an eight-dimensional vector
space with basis $\{[], [1], [2], [3], [12], [13], [23], [123]\}$.  A
typical element of $\ga(3)$ is then equal to $s[] + a[1] + b[2] + c[3]
+ A[23] + B[13] + C[12] + S[123]$ where $s, a, b, c, A, B, C,$ and $S$
are real numbers.

We consider elements of $\ga(3)$ of the form $a[1]+b[2]+c[3]$ to be
vectors (or rank one elements) and this three-dimensional vector space
is the geometric entity $\ga(3)$ models.  Elements of the form $s[]$,
also just written $s$, are scalars (or rank zero elements), so in a
geometric algebra, one can add a scalar to a vector.  Elements of the
form $A[23] + B[13] + C[12]$ are bivectors (rank two elements), which
model oriented plane regions with an area magnitude.

Just as a vector has direction and magnitude, and one considers two
vectors having the same direction and magnitude to be equal even if
they have different locations in three-space, a bivector has a
containing plane, one of two orientations on that plane, and an area,
so two planar regions having the same area, contained in parallel or
equal planes, and given the same orientation (which may be thought of
as clockwise versus counterclockwise) are considered to be the same
bivector.

Elements of the form $S[123]$ are considered volume elements (rank
three elements), also called pseudoscalars because they have one
degree of freedom in three-dimensional space.  A volume element only
has magnitude and orientation (positive or negative, which may be
thought of as being a right-handed or left-handed coordinate system).

Addition of two elements expressed in the linear combination form
mentioned above is done by combining like terms.  If $x = 4 + 3[1] +
4[2] + 5[12]$ and $y = 3 + 2[1] + 3[2] + 4[12]$, then
$x+y = 7 +5[1] +7[2] + 9[12]$.

Multiplication works like polynomial multiplication, except without
commutativity, and using the rules of converting products of blades
into the standard ones $[]=1$, $[1]$, $[2]$, $[12]$, $[3]$, $[13]$,
$[23]$, and $[123]$.  So, using the same $x$ and $y$ as above, to find
$xy$, one multiplies every term of $x$ by every term of $y$, then
reduces and combines like terms:
\begin{alignat*}{4}
  xy &=&&12 + 8[1] + 12[2] +16[12]\\
  && +~ & 9[1] + 6[1][1] + 9[1][2] + 12[1][12]\\
  && +~ & 12[2] + 8[2][1] + 12[2][2] + 16[2][12]\\
  && +~ & 15[12] + 10[12][1] + 15[12][2] + 20[12][12]\\
  &=&&12 + 8[1] + 12[2] + 16[12]\\
  && +~ &9[1] + 6 + 9[12] + 12[2]\\
  && +~ & 12[2] - 8[12] + 12 - 16[1]\\
  && +~ & 15[12] - 10[2] + 15[1] - 20\\
  &=&& 10 + 16[1] + 26[2] + 32[12]
\end{alignat*}
\end{example}

\subsection{Standard Basis Blades}

Generalizing from the example, we see that swapping adjacent unequal
indices of a basis blade changes its sign and deleting pairs of
adjacent, equal indices does not change the element.  Thus, all basis
blades in $\ga(n)$ are either plus or minus a \term{standard} basis
blade $[i_1i_2\cdots{i}_m]$ ($m \le n$) in which the indices are in
increasing order and there are no duplicate indices.  In this case,
the \term{rank} of the basis blade is $m$.  Any element of $\ga(n)$
can be written uniquely (up to ordering of terms) as a linear
combination of standard basis blades (including $[]$, the rank-zero
empty basis blade equal to $1$ in the algebra), and the rank of that
element is the maximum rank among the basis blades in the linear
combination.  The rank of the $0$ element is defined to be $-\infty$.
Thus, $\ga(n)$ is an algebra filtered by rank.

A nonzero element of $\ga(n)$ that is a linear combination of standard
basis blades of the same rank $r$ is said to be \term{purely} of rank
$r$.  Rank 0 elements (that is, pure rank 0 elements along with the 0
element) are called \term{scalars}.  Pure rank 1 elements (along with
0) are called \term{vectors}.  Pure rank two elements (along with 0)
are called \term{bivectors}, and so on.  All elements are called
\term{multivectors}. When the ambient rank is $n$, rank $n$ elements
(along with 0) are called \term{pseudoscalars}, rank $n-1$ elements
(along with 0) are called \term{pseudovectors}, and so on.

The algebra $\ga(n)$ is generated, as a vector space, by the standard
basis blades $[]$, $[1]$, $[2]$, $\dots$, $[n]$, $[12]$, $\dots$
$[123\dots{n}]$, of which there are $2^n$, so as a real vector
space, $\ga(n)$ has dimension $2^n$.

In addition to the geometric product, other products are defined in
$\ga(n)$.  The \term{inner product} is the symmetric component of the
geometric product: $a\cdot{b} = \frac{ab+ba}{2}$.  The \term{outer
  product} is the antisymmetric component of the geometric product:
$a\wedge{b}=\frac{ab-ba}{2}$.  As the names imply,
$a\cdot{b}=b\cdot{a}$, and $a\wedge{b} = -b\wedge{a}$.  Then, $ab =
a\cdot{b} + a\wedge{b}$.

\subsection{Additional Operations}

Note that for vectors, $a\cdot{b}$ is exactly the standard dot
product, using $\{[1], [2], \dots, [n]\}$ as the standard ordered
basis.  For vectors in $\ga(3)$, the wedge product has the same
magnitude as the cross product, but is a bivector (pseudovector)
rather than a vector.  Also, the wedge product is defined in any
number of dimensions, not just three, and for non-vector elements as
well.

The \emph{standard pseudoscalar} $I$ of $\ga(n)$ is defined to be the
rank-$n$ standard basis blade $[123\dots{n}]$.  Then, $I^2 = II$ must
either be $-1$ or $1$, so $I^4=1$ regardless of $n$.  In particular,
the subalgebra of $\ga(2)$ generated by $[]$ and $I$ is isomorphic to
the complex number field.  In fact $I^2=-1$ in $\ga(2)$ and the
isomorphism maps $1$ to $[]$ and $\imath$ to $I$.  In addition,
$\ga(3)$ contains a subalgebra isomorphic to the quanternion algebra.
This subalgebra is generated by $[]$, $i=[23]$, $j=[13]$, and
$k=[12]$, with $[]$ maping to 1 and $i$, $j$, and $k$ mapping to the
same-named elements of the quaternion ring.

The \term{norm} $|a|$ of an element $a\in\ga(n)$ is defined to be the
square root of the sum of the squares of the coefficients of the
standard basis blades making up the linear combination.  Note for
vectors this reduces to the usual norm for vectors.

The \term{reversion} $\tilde{a}$ of an element $a\in \ga(n)$ is formed
from $a$ by taking the linear combination of standard basis blades
composing $a$, and then reversing the order of multiplication of the
algebra generators in each term: $[i_{j_1}][i_{j_2}]\cdots[i_{j_m}]$
$\rightarrow$ $[i_{j_m}][i_{j_{m-1}}]\cdots[i_{j_1}]$.  Equivalently,
each term of $a$ of pure rank $r$, where $r$ is congruent to $2 $ or
$3$ modulo $4$, is negated.

The \term{grade projection} $\mathrm{proj}_k(a)$ of an element $a\in \ga(n)$ is
formed from $a$ by deleting all terms of pure rank not equal to $k$.

The \term{left inverse} of $a\in \ga(n)$ is an element $b$ satisfying
$ba=1=[]$ and the \term{right inverse} $c$ likewise satisfies $ac=1$.
We use $a^{-1}$ to represent an element that is both the left and right
inverse of $a$.  If $a{\tilde{a}}$ is a nonzero scalar, then
$\frac{a}{a{\tilde{a}}}$ is well-defined by dividing each coefficient of
$a$ by the scalar denominator.  In such a case, $a$ does have a
two-sided inverse and $a^{-1} = \frac{a}{a{\tilde{a}}}$.

The \term{dual} of an element $a$ of $\ga(n)$ is $aI^{-1}$.  Note that
$I^{-1}$ exists and is equal to $I^3$, which is either $I$ or $-I$.
Note that the dual of a scalar is a psuedoscalar, the dual of a vector
is a pseudovector, and so on.  The double dual of $a$ is either $a$ or
$-a$.

Note that in $\ga(3)$, $a\wedge{b}I^{-1}$ is a vector, and is equal to
the vector space cross product $a\times{b}$.  Thus, we define a
\term{generalized cross product} on any $\ga(n)$ to be
$a\times{b}=a\wedge{b}I^{-1}$, the dual of the wedge product.  In
dimensions other than 3, the cross product of vectors is no longer a
vector.  However, in $n$ dimensions, one can take the dual of the
wedge product of $n-1$ vectors to give a vector which geometrically is
similar to the cross product in three dimensions.

The \term{exponential} of $a\in{\ga}(n)$ is computed as follows.  If
$a=0$, $e^0=1$ by definition.  Otherwise, let $\theta=|a|$ and
$b=a/\theta$.  Then use DeMoivre's formula with $b$ taking the role of
the imaginary unit: $e^a = \cos\theta + b\sin\theta$.  The inverse of
this operation, a geometric algebraic logarithm, is difficult and
multivalued, even more so than with complex numbers.

The main use for the exponential is in computing a \term{rotor}.  If
$v$ and $w$ are linearly-independent unit vectors (that is,
$|v|=|w|=1$) in the $n$-dimensional vector space whose basis is the
rank-one standard blades, then $vw$ is a bivector that represents an
oriented plane in the $n$-dimensional space.  A rotation about the
origin in the direction from $v$ to $w$ along an angle $theta$ is then
modeled by the \term{rotor} $r$ in $\ga(n)$ that is $r =
e^{\frac12vw\theta}$.  Then if $a$ is any element (vector or
otherwise) of $\ga(n)$, its rotation is then $ra{\tilde{r}}$.  If $a$
is a vector, so is $ra{\tilde{r}}$, and this is an ordinary vector
rotation.  If $a$ is a bivector representing an oriented plane,
$ra{\tilde{r}}$ is the rotation of the plane.  And so on.

The \term{meet} of two elements $a$ and $b$ in $\ga(n)$ is defined by
$aI^{-1}b$, or the product of the dual of $a$ with $b$.

\section{Algorithms}
\newcommand{\n}{\ensuremath{\mathop{\mathrm{N}}}} An apparent way to
represent a standard basis blade $a=[i_1i_2\dots{i}_m]$ is via the
integer $\n(a)$ defined to be $\sum_{k=1}^{m}2^{i_k}$.  Then, the
product $ab$ of $a=[i_1i_2\dots{i}_m]$ and $b=[j_1j_2\dots{j}_r]$
can be computed as follows:

\begin{enumerate}
\item If $m=0$, let $c$ be the multivector $b$ and return $c$
\item If $r=0$, let $c$ be the multivector $a$ and return $c$
\item Let $c$ be the blade for which $\n(c)$ is the bitwise exclusive
  or of $\n(a)$ and $\n(b)$.
\item Let $s$ be 1.
\item Let $p$ be the maximum of $\n(a)$ and $\n(b)$.
\item Let $d$ be $m$
\item Let $e$ be $1$
\item while $e\le{p}$, do the following:
  \begin{enumerate}
  \item Compute the bitwise and of $e$ and $\n(a)$.
  \item If the result of the bitwise and is nonzero, then replace $d$ with $d-1$.
  \item If $d$ is odd, Compute the bitwise and of $e$ and $\n(b)$.
  \item If $d$ is odd and the result of the bitwise and is nonzero
    then replace $s$ with $-s$.
  \item Replace $e$ with $2e$.
  \end{enumerate}
\item Let $c$ be the multivector that is the real coefficient $s$
  multiplied by the blade given by $c$, and return $c$.
\end{enumerate}

A standard basis blade is simply a nonnegative integer in this
representation.  A choice needs to be made to represent a multivector,
a real linear combination of these blades.

The \term{dense} representation of a multivector in $\ga(n)$ is an
array $A$ of dimension $2^n$, thus having elements $A_0$, $A_1$, and
so on up to $A_{2^n-1}$.  Then, a multivector $\sum_{i=1}^{l}c_ib_i$,
where the $c_i$ are real numbrers and the $b_i$ are standard basis
blades in $\ga(n)$, is represented by the array $A$ in which for
$1\le{i}\le{l}$, $A_{\n(b_i)} = c_i$, and all other elements of $A$
are zero.

The dense representation requires little computation, but uses
$O(2^n)$ memory per element of $\ga(n)$, so it is useful for low
dimensionality, such as in representing the commonly-used $\ga(3)$.

For larger dimensions, a \term{sparse} representation is needed.  Let
$D$ be a \term{dictionary}, algorithmically represented by a hash
table in most cases, but mathematically works like the array: for any
nonnegative integer $i$, $D_i$ is a real number.  For most $i$, this
real number is zero and thus $i$ has no entry in the hash table.  Only
when $D_i$ is nonzero is there an entry in the hash table with key $i$
and value $D_i$. 
  
Then, a multivector $\sum_{i=1}^{l}c_ib_i$, where the $c_i$ are real
numbrers and the $b_i$ are standard basis blades in $\ga(n)$, is
represented by the dictionary $D$ in which for $1\le{i}\le{l}$,
$D_{\n(b_i)} = c_i$, and no other entries are present in $D$ (so that
the dictionary returns the default value of 0 for those).

The sparse representation requires a small amount of computation for
hashing and, sometimes, rebuilding the table, but only requires $O(l)$
memory for an element of $\ga(n)$ having $l$ nonzero terms.  Thus, the
sparse representation is most useful for large $n$ when many elements
are such that most standard basis blades have a zero coefficient.

In both cases, the addition procedure is evident and involves adding
the values from one array or dictionary to the same-indexed values of
the other.

The geometric product, however, requires an all-to-all mapping of
array or dictionary elements.  We illustrate the algorithm for the
array case, with the dictionary case being nearly identical, just
iterating over keys in the hash table instead of over all integers in
the range from $0$ to $2^{n-1}$.

Suppose $A$ and $B$ are arrays representing multivectors in $\ga(n)$.
We compute a new array $C$ with $2^n$ entries representing the
geometric product of the multivectors $A$ and $B$.

\begin{enumerate}
\item Initialize $C$ to $C_0=C_1=\cdots=c_{2^{n-1}}=0$.
\item For each $i$ from 0 to $2^{n-1}$ such that the coefficient $A_i$
  is nonzero
  \begin{enumerate}
  \item For each $j$ from 0 to $2^{n-1}$ such that the coefficient
    $B_j$ is nonzero
    \begin{enumerate}
      \item Let $k$ be the integer that is the result of multiplying
        together the blades represented by $i$ and $j$, and let $s$ be
        the coefficient, $1$ or $-1$, of the result.
      \item Let $C_k$ be $C_k + A_iB_js$, an ordinary sum and product
        of real numbers.
    \end{enumerate}
  \end{enumerate}
  \item Return the multivector $C$.
\end{enumerate}

\bibliographystyle{amsplain}

\end{document}
